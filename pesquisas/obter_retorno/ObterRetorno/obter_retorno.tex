\documentclass[14pt, a4paper]{article}
\usepackage{xcolor}
% Definindo novas cores
\definecolor{verde}{rgb}{0.25,0.5,0.35}
\definecolor{jpurple}{rgb}{0.5,0,0.35}
% Configurando layout para mostrar codigos Java
\usepackage{listings}
\lstset{
  language=Java,
  basicstyle=\ttfamily\small, 
  keywordstyle=\color{jpurple}\bfseries,
  stringstyle=\color{red},
  commentstyle=\color{verde},
  morecomment=[s][\color{blue}]{/**}{*/},
  extendedchars=true, 
  showspaces=false, 
  showstringspaces=false, 
  numbers=left,
  numberstyle=\tiny,
  breaklines=true, 
  backgroundcolor=\color{cyan!10}, 
  breakautoindent=true, 
  captionpos=b,
  xleftmargin=0pt,
  tabsize=4
}
\pagestyle{empty}
\usepackage[utf8]{inputenc}
\usepackage{amsmath}
\usepackage{amsfonts}
\usepackage{amssymb}
\usepackage[brazil]{babel}
\usepackage{indentfirst}
\usepackage{url}

\RequirePackage{graphicx}
\title{Class Runtime}
\author{Andrey Silva Ribeiro \and Jônatas de Souza Rezende \and Lucas Pereira de Azevedo}
\usepackage[left=3cm,right=3cm,top=2cm,bottom=2cm]{geometry}

\begin{document}

\begin{titlepage}


\begin{center}
\begin{figure}[htb]
		
		\label{figura:LogoIF}
	
		\centering
		\includegraphics[width=6cm]{logo.png} 
\end{figure}


Instituto Federal Goiano - Campus Ceres\\
Bacharelado em Sistemas de Informação\\
Prof. Me. Ronneesley Moura Teles\\\vspace{0.2cm}
Andrey Silva Ribeiro \\
Jônatas de Souza Rezende \\
Lucas Pereira de Azevedo \\\vspace{7.0cm}

\textit{\textbf{\Large{Obter retorno através do comando}}}\\\vspace{0.5cm}
\textit{\textbf{\Large{Runtime.getRuntime().exec("comando")}}}\\\vspace{9.5cm}

Outubro\\
2017\\
\end{center}
\end{titlepage}

\tableofcontents

\newpage
\begin{center}
\textbf{\Large{Runtime.getRuntime().exec("comando")}}\\\vspace{0.5cm}
\end{center}
\section{Como funciona?}%GGN

A classe \textbf{Runtime} possui um método para fazer uma chamada ao sistema operacional e executar algum programa, que é o método \textbf{getRuntime().exec("comando")}. A mesma deve ser utilizada em conjunto com a classe \textbf{Process}.

Aconselha-se que seja pouco utilizado, pois ele gera uma dependência em relação ao sistema operacional em questão, perdendo a portabilidade.

\section{Código exemplo - Java.}%GGN

\begin{lstlisting}
import java.io.BufferedReader;
import java.io.File;
import java.io.FileReader;
import java.io.FileWriter;
import java.io.IOException;
import java.io.InputStreamReader;

public class Pegar_retorno {

    public static void main(String[] args) throws Exception {
        
        try{

            String resultado; 
            
            String n_arquivo = "resultado.txt";
            
            String comando = "CMD /C Rscript d:\\desvio_padrao.R";
            
            Process exec = Runtime.getRuntime().exec(comando);
        
            BufferedReader retorno = new BufferedReader(new InputStreamReader(exec.getInputStream()));
          
            while ((resultado = retorno.readLine()) != null){
                FileWriter arquivo;
				try {
                    arquivo = new FileWriter(new File(n_arquivo));
                    arquivo.write("Resultado da operacao: ");
                    arquivo.write(resultado);
                    arquivo.close();
                    FileReader reader = new FileReader(n_arquivo);
                    BufferedReader leitor = new BufferedReader(reader);
                    while(leitor.ready()) {
                            System.out.println(leitor.readLine());
                    }
                    reader.close();
                    leitor.close();
                    
                } catch (IOException e) {
                    System.err.println("Erro: " + e.toString());
				} catch (Exception e) {
                    System.err.println("Erro: " + e.toString());
				}
            }
            
            System.out.println("\nResultado gravado em resultado.txt\n");
            retorno.close();
            exec.destroy();
        }
        
        catch(IOException exc){
            System.err.println("Erro: " + exc.toString());
        }
    }        
}
\end{lstlisting}

\section{Código exemplo - R}

Abaixo segue o código contido no script do \textbf{software R} que será usado no comando para obtenção do retorno:

\begin{lstlisting}
x <- c(65, 72, 70, 72, 60, 67, 69, 68)
 
sd(x)
\end{lstlisting}

Executando o script acima o resultado apresentado é o seguinte:

\begin{lstlisting}
[1] 3.97986
\end{lstlisting}

\section{Explicação do código.}

\begin{enumerate}
\item Importações necessárias para a execução do código Java:

\begin{lstlisting}
import java.io.BufferedReader;
import java.io.File;
import java.io.FileReader;
import java.io.FileWriter;
import java.io.IOException;
import java.io.InputStreamReader;
\end{lstlisting}

\item Declaração das variáveis.

\begin{lstlisting}
String resultado; 
String n_arquivo = "resultado.txt";
String comando = "CMD /C Rscript d:\\desvio_padrao.R";
\end{lstlisting}

\item O método \textbf{exec} retorna um \textbf{Process} onde é possível obter a saída do programa e enviar dados para entrada, dentre outros.

\begin{lstlisting}
Process exec = Runtime.getRuntime().exec(comando);
\end{lstlisting}

\item A classe \textbf{BufferedReader} serve para leitura de uma \textbf{InputStreamReader}, que por sua vez lê bytes e decodifica-os em caracteres, esses dados serão armazenados na variável \textbf{retorno}.
        
\begin{lstlisting}
BufferedReader retorno = new BufferedReader(new InputStreamReader(exec.getInputStream()));
\end{lstlisting}

\item Se o retorno do método \textbf{exec} for diferente de nulo, a variável \textbf{resultado} recebe o valor que foi retornado, e a classe \textbf{FileWriter} que escreve dados em arquivos, pega esse retorno e salva no arquivo.

\begin{lstlisting}          
while ((resultado = retorno.readLine()) != null){
	FileWriter arquivo;
\end{lstlisting}                

\item Especifica o que vai ser escrito no arquivo de destino, ainda dentro da condicional. Na linha 1 será criado ou usado o arquivo, na linha 2 o texto que será escrito no arquivo, na linha 3 o resultado obtido a ser salvo no arquivo e na linha 4 finaliza o arquivo.

\begin{lstlisting}
arquivo = new FileWriter(new File(n_arquivo));
arquivo.write("Resultado da operacao: ");
arquivo.write(resultado);
arquivo.close();
\end{lstlisting}                    

\item A classe \textbf{FileReader} executa a leitura dos dados do arquivo, cria a variável \textbf{reader} e armazena os dados obtidos na variável criada.

\begin{lstlisting}                    
FileReader reader = new FileReader(n_arquivo);
\end{lstlisting}                                        

\item A classe \textbf{BufferedReader} instancia a variável \textbf{leitor} onde armazena os dados lidos anteriormente. Na linha 3 é impresso na tela os valores obtidos enquanto tiver dados armazenados, na linha 5 encerra a variável \textbf{reader} e na linha 6 encerra a variável \textbf{leitor}.

\begin{lstlisting}
BufferedReader leitor = new BufferedReader(reader);
while(leitor.ready()) {
	System.out.println(leitor.readLine());
}
reader.close();
leitor.close();
\end{lstlisting}                                        

\item Após toda a execução do código, apresenta na tela informando o nome do arquivo em que foi gravado os dados. Na linha 2 finaliza a variável \textbf{retorno} e na linha 3 encerra o método \textbf{exec}.

\begin{lstlisting}                    
System.out.println("\nResultado gravado em resultado.txt\n");
retorno.close();
exec.destroy();
\end{lstlisting}                    

\end{enumerate}

\section{Referências}
\begin{enumerate}

\item ORACLE. \textbf{Docs Oracle.} Disponível em: $<$https://docs.oracle.com/javase/7/docs/api/allclasses-noframe.html$>$. Acesso em: 5 de outubro de 2017.

\end{enumerate}

\end{document}


\end{document}